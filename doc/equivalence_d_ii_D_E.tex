% Preamble {{{1
\documentclass[A4paper, 12pt]{article}

\usepackage[]{amsmath}
\usepackage[]{hyperref}
\usepackage[utf8]{inputenc}
\usepackage[margin=0.5cm]{geometry}

% Spins
\newcommand{\uu}{\left | \uparrow\uparrow \right \rangle}
\newcommand{\dd}{\left | \downarrow\downarrow \right \rangle}
\newcommand{\du}{\left | \downarrow\uparrow \right \rangle}
\newcommand{\ud}{\left | \uparrow\downarrow \right \rangle}
\newcommand{\mone}{\frac{1}{\sqrt{2}}\left( \ud + \du\right)}
\newcommand{\mmone}{\frac{1}{\sqrt{2}}\left( \ud - \du\right)}
\newcommand{\x}{\frac{1}{\sqrt 2}\left( \uu + \dd\right)}
\newcommand{\y}{\frac{1}{\sqrt 2}\left( \uu - \dd\right)}

\begin{document}

%\tableofcontents

\section{Introduction} %{{{1
\label{sec:introduction}


This is the proof of the equivalence of the two representations of the zero
field spin-spin Hamiltonian. \\

One can represent the spin-spin as

\begin{equation}
  \hat{H} _{ss} =
  + D _{x} \hat S ^{2} _{x}
  + D _{y} \hat S ^{2} _{y}
  + D _{z} \hat S ^{2} _{z}
\end{equation}

This operator is in this case represented in terms of the spin component
operators $ \hat{S}_{i}  $ where $ i \in \left \{ x,y,z \right \} $.  It turns
out to be quite useful to represent the same operator in terms of the operators
$ \hat{S}_{+}^{2}   $, $ \hat{S}_{-}^{2}   $, $ \hat{S}_{z}^{2}   $ and $
\mathbf{S}^{2}  $.  From this alternative representation new constants $ E $
and $ D $ appear.\\

In the first representation, one can write the coefficients $ D_{i}  $  in matrix form,

\begin{equation}
  \mathsf{D} =
  \left(
    \begin{matrix}
      D _{x} &         &         \\
              & D _{y} &         \\
              &         & D _{z}
    \end{matrix}
  \right)
\end{equation}

This matrix is traceless, i.e. $ \sum_{i} D _{ii} = 0
$. The connection between coefficients $ E  $ and $ D $ is

\begin{equation}
  E = \frac{D _{y} - D _{x}}{2}, \quad D = \frac{3}{2}D _{z}
\end{equation}

as it will be shown in the coming section.  A very quick calculation shows that

\begin{equation}
  \left\{
  \begin{matrix}
   D _{x}&=&-E-\frac{1}{3} D \\
                             \\
   D _{y}&=&E-\frac{1}{3}  D \\
                             \\
   D _{z}&=&\frac{2}{3}    D
  \end{matrix}
  \right .
\end{equation}

which means that the diagonal elements are obtainable from the coefficients $ E
$ and $ D $ under the assumption that the matrix $ \mathsf{D} $ is traceless.










\section{Formula deduction} %{{{1
\label{sec:formula_deduction}

We part from the following Hamiltonian:

\begin{equation}
  \hat{H} _{ss} =
  + D _{x} \hat S ^{2} _{x}
  + D _{y} \hat S ^{2} _{y}
  + D _{z} \hat S ^{2} _{z}
\end{equation}

and for general angular momentum operators we have the following properties and
definitions:

\begin{align}
  \hat{S} _{\pm}                               & = \hat{S} _{x} \pm i \hat{S} _{y}\\
  \left\{\hat{S} _{+} , \hat{S} _{-}  \right\} & = 2(\mathbf{S} ^{2} - \hat{S} _{z} ^{2} )
\end{align}

Let us replace in the expression for $ \hat{H} _{ss} $ the coefficients $ D
_{i}  $ by $ E $ and $ D $ , i.e.:


\begin{align*}
  \hat{H} _{ss} &=
  - \left(E + \frac{1}{3}D\right)  \hat S ^{2} _{x}
  + \left(E - \frac{1}{3}D\right)  \hat S ^{2} _{y}
  + \frac{2}{3}D \hat S ^{2} _{z} \\
  &=
  E \left( \hat{S} _{y} ^{2}  -\hat{S} _{x} ^{2}  \right)
  +
  \frac{D}{3} \left[ -\hat{S} ^{2} _{x} -\hat{S} _{y} ^{2} + 2\hat{S} _{z} ^{2}  \right]
\end{align*}

now we can express of course

\begin{align*}
  -\hat{S} ^{2} _{x} -\hat{S} _{y} ^{2} + 2\hat{S} _{z} ^{2} &=
  \hat{S} ^{2} _{z} -\mathbf S^{2} + 2\hat{S} _{z} ^{2} =
  3\hat{S} ^{2} _{z} -\mathbf S^{2}
\end{align*}

With this we can express the part of $ \hat{H} _{ss}   $ depending on $ D $ as

\begin{equation}
  D
  \left(
    \hat{S} _{z} ^{2} - \frac{1}{3} \mathbf{S} ^{2}
  \right).
\end{equation}

On the other hand we have the term $ \hat{S} _{y} ^{2} -\hat{S} _{x} ^{2}  $
which we can rewrite as

\begin{equation}
  \mathbf{S} ^{2} - \hat{S}_{z}^{2} - 2 \hat{S}_{x}^{2}
  =
  \frac{1}{2}\left\{\hat{S} _{+} , \hat{S} _{-}  \right\}
  - \frac{2}{4}\left( \hat{S} _{+}  + \hat{S} _{-}   \right) ^{2}
  =
  -\frac{\hat{S} _{+} ^{2} + \hat{S} _{-} ^{2}}{2}
\end{equation}

With these small calculations we can write for the spin-spin Hamilton operator

\begin{equation}
  \label{eq:spin-spin-hamiltonian}
  \hat{H} _{ss}
  =
  \frac{-E}{2}\left( \hat{S} _{+} ^{2} + \hat{S} _{-} ^{2}   \right)
  +
  D
  \left(
    \hat{S} _{z} ^{2} - \frac{1}{3} \mathbf{S}^{2}
  \right).
\end{equation}





\section{Eigenvectors} %{{{1
\label{sec:eigenvectors}

The eigenvectors of the operator $ \hat{H} _{ss}   $ are not the common
eigenbasis of $ \hat{S}_{z} $ and $ \mathbf{S}^{2}  $, i.e.,

\begin{equation}
  \left\{
    \uu, \dd, \mone, \mmone
  \right\}
\end{equation}

since the operators $ \hat{S}_{\pm}  $ appear in the expression through the
constant $ E $. In particular, whenever $ E=0 $ the usual eigenbasis of $
\hat{S}_{z}  $ and $ \mathbf{S}^{2}  $ is the solutions to the problem,
therefore the quantum numbers $ m $ and $ j $ are good.\\

%This is so because of the states with quantum number $ m_S \neq 0 $ , for
%example one sees that
%\begin{equation}
  %\hat{H} _{ss} \uu = a\uu + b \dd
%\end{equation}
%with $ a $ and $ b $ different from zero.

In general $ E\neq 0 $, the new eigenbasis is formed by the following set of vectors:

\begin{equation}
  \label{eq:basis-of-spin-spin-hamiltonian}
  \left\{\x, \y,\mone,  \mmone \right\}
\end{equation}
The effect of $ \hat{H}_{ss}  $ is therefore mixing the vectors with $ m=-1 $
($ \dd $)  and $ m=1 $  ($ \uu $).  In order to see that this is indeed the
case, one has to know how each operator appearing in $ \hat{H} _{ss}  $  acts
on every state.\\

In general we have for an angular momentum operator $ \hat{J}  $ that

\begin{equation}
  \hat{J} _{\pm} \left | j, m \right \rangle = c _{\pm} \left | j,m \pm 1 \right \rangle
\end{equation}

where

\begin{equation}
  c _{\pm}  = \hbar \sqrt{(j\mp m)(j \pm m +1)} = \hbar \sqrt{j (j+1) - m (m\pm 1) }
\end{equation}

where $ j $ and $ m $ are the pertinent quantum numbers. In our case $ j \in
\left \{0,1 \right \}  $.  We can calculate the eigenvalues for $ \uu $ and $
\dd $, we need specially the following result

\begin{align*}
  \hat{S} _{-} ^{2} \uu &=
  \hat{S} _{-} ^{2} \left | j=1, m=1 \right \rangle =
  \sqrt{2\cdot 1} \hbar \hat{S} _{-}  \left | j=1, m=0 \right \rangle = \\
  &
  \sqrt{1\cdot 2} \hbar  \sqrt{2\cdot 1} \hbar  \left | j=1, m=-1 \right \rangle =
  2\hbar ^{2} \dd
\end{align*}

and in the same spirit one can calculate $ \hat{S} _{+} ^{2} \dd = 2 \hbar ^{2} \uu  $.
Now we can calculate the eigenvalues:

\begin{equation}
  \hat{H} _{ss} \left( \uu \pm \dd \right)
  =
  \left(
    \frac{D}{3}\pm (-E)\hbar
  \right)
  \left(
    \uu \pm \dd
  \right)
\end{equation}

Since $ \mmone $ has quantum numbers $ j=0 $ and $ m=0 $ we have naturally

\begin{equation*}
  \hat{H} _{ss} \left( \mmone \right) = 0
\end{equation*}

and of course for the case $ j=1 $ and $ m=0 $ we have

\begin{equation}
  \hat{H} _{ss} \left( \mone \right) = -\frac{2}{3}D \left( \mone \right)
\end{equation}

Therefore, in this

\begin{equation}
  \begin{pmatrix}
    \frac{D}{3} - E &                 &               &   \\
                    & \frac{D}{3} + E &               &   \\
                    &                 & -\frac{2}{3}D &   \\
                    &                 &               & 0
  \end{pmatrix}
  \begin{matrix}
    \x     \\
    \y     \\
    \mone  \\
    \mmone
  \end{matrix}
\end{equation}





\appendix
\section{Eigenvalues calculation} %{{{1
\label{sec:eigenvalues_calculation}

Let us consider a general eigenvector
$ \left | j,m \right \rangle  $. We know how $ \hat{S}_{z}  $ and $ \mathbf{S}^{2}  $ act on it, i.e.
\begin{align*}
\hat{S}_{z}^{2}  \left | j,m \right \rangle        & =
(\hbar m)^{2} \left | j,m \right \rangle        \\
\mathbf{S}^{2}  \left | j,m \right \rangle         & =
\hbar^{2}  j(j+1)\left | j,m \right \rangle   \\
\end{align*}

We can perform the same calculation for $ \hat{S}_{\pm}^{2}  $, this goes as

\begin{align*}
\hat{S}_{\pm}^{2} \left | j,m \right \rangle &=
  \hbar\sqrt{j(j+1)- m(m\pm 1) } \hat{S}_{\pm} \left | j,m\pm 1 \right \rangle \\
&=
  \hbar^{2}\sqrt{j(j+1)- m(m\pm 1) }\sqrt{j(j+1)- m(m\pm 2) } \left | j,m\pm 2 \right \rangle \\
\end{align*}

Let us remember the form of the Hamiltonian $ \hat{H}_{ss} $ in equation \ref{eq:spin-spin-hamiltonian}

\begin{equation*}
  \hat{H} _{ss}
  =
  \frac{-E}{2}\left( \hat{S} _{+} ^{2} + \hat{S} _{-} ^{2}   \right)
  +
  D
  \left(
    \hat{S} _{z} ^{2} - \frac{1}{3} \mathbf{S}^{2}
  \right)
  = \hat{H}_{ss}(E) + \hat{H}_{ss}(D)
\end{equation*}

The general basis we have built is formed by expressions of the form $ \left |
\psi \right \rangle = c_{1} \left | j_{1},m_{1} \right \rangle  + c_{2} \left |
j_{2}, m_{2} \right \rangle $. Let us calculate the general form of $
\hat{H}_{ss} \left | \psi  \right \rangle   $ and afterwards substitute the
particular values for the members of the basis. In this context, one sees that

\begin{align}
  \label{eq:D_part_spin_spin_general_calculation}
\hat{H}_{ss}(D) \left | \psi  \right \rangle &=
  \hbar^{2}D
  \left(
    c_{1}\left(m_{1}^{2} - \frac{1}{3} j_{1}(j_{1}+1)\right)
    \left | j_{1}, m_{1} \right \rangle +
    c_{2}\left(m_{2}^{2} - \frac{1}{3} j_{2}(j_{2}+1)\right)
    \left | j_{2}, m_{2} \right \rangle
  \right)
\end{align}

Analogously we have for the part of $ \hat{H}_{ss} $ depending on $ E $ that

\begin{align}
  \label{eq:E_part_spin_spin_general_calculation}
\hat{H}_{ss}(E) \left | \psi  \right \rangle &=
  \frac{-E\hbar^{2}}{2}
  \sum^{2}_{l=1}
  \sum^{1}_{k=0}
  c_{l}
  \sqrt{j_{l}(j_{l}+1)- m_{l}(m_{l}+(-1)^{k} 1) }\sqrt{j_{l}(j_{l}+1)- m_{l}(m_{l}+(-1)^{k} 2) }
\left | j_{l},m_{l}+(-1)^{k} 2 \right \rangle
\end{align}

where $ k $ just accounts for the sign that arises in considering $ \hat{S}_{+}
$ or $ \hat{S}_{-} $ and $ l $ accounts for both parts of $ \left | \psi
\right \rangle  $.
Now we can see which is the outcome for the basis provided in equation
\ref{eq:basis-of-spin-spin-hamiltonian}.

\begin{description}
  \item[$ \left | \psi  \right \rangle $]
  \item[$\x$] In this case we have $ m_{1} = 1 $, $ m_{2} = -1 $, $ j_{1}=j_{2}=1 $, $ c_{1} = c_{2} = \frac{1}{ \sqrt{2 } } $.
    From equation \ref{eq:D_part_spin_spin_general_calculation} we can calculate therefore
    \begin{align*}
    \hat{H}_{ss}(D) \left | \psi  \right \rangle &=
      \hbar^{2}D
      2\left(1- \frac{2}{3}\right)
    \left | \psi  \right \rangle =
    \frac{2}{3}
    \hbar^{2}D
  \left | \psi  \right \rangle
\end{align*}

In the same way we can calculate the action of $ \hat{H}_{ss}(E) $ following
equation \ref{eq:E_part_spin_spin_general_calculation}.  In the summation note
that some terms will disappear since $ m_{l} \in \left \{ -1,1 \right \}  $ and
$ j_{l} = 1 $, so for $ l=1 $ we will only have a part different from zero for
$ k=1 $ since $ m_{1} = 1 $. In the same way we will have only a nonzero part
for $ l=2 $ whenever $ k = 0 $.

\begin{align*}
\hat{H}_{ss}(E) \left | \psi  \right \rangle
&=
\frac{-E\hbar^{2}}{2}
\frac{1}{ \sqrt{2 } }
\sqrt{2- m_{1}(m_{1}-1) }
\sqrt{2- m_{1}(m_{1}-2) }
  \left | 1,m_{1}-2 \right \rangle \\
  &+
  \frac{-E\hbar^{2}}{2}
  \frac{1}{ \sqrt{2 } }
  \sqrt{2- m_{2}(m_{2}+ 1) }
  \sqrt{2- m_{2}(m_{2}+ 2) }
\left | 1,m_{2}+2 \right \rangle \\
&=
\frac{-E\hbar^{2}}{2}
\frac{1}{ \sqrt{2 } }
\sqrt{2}
\sqrt{3}
  \left | 1,-1\right \rangle
  +
  \frac{-E\hbar^{2}}{2}
  \frac{1}{ \sqrt{2 } }
  \sqrt{2}
  \sqrt{3}
\left | 1,1\right \rangle \\
&=
-\frac{ \sqrt{3 } }{ \sqrt{2 } }E\hbar^{2}
  \left | \psi  \right \rangle
\end{align*}

Therefore we have for this case
\begin{equation}
  \label{eq:x-vector-eigenvalue-equation}
  \hat{H}_{ss}
\left | \psi  \right \rangle
=
\hbar^{2}
\left(
  \frac{2}{3} D
  -
  \frac{\sqrt{3}}{\sqrt{2}}E
\right)
\left | \psi  \right \rangle
\end{equation}

  \item[$\y$] The same argument and calculation is here valid as above, the
    only difference is in the minus sign before $ \dd = \left | j=1,m=-1 \right
  \rangle  $. Therefore the $ D $ part remains untouched. In the term depending
  on $ E $, since $ \hat{H}_{ss}(E) $ has the effect of swapping $ m=-1 $ into
  $ m=1 $, the only difference with above will be a minus sign, therefore in
  this case we have a similar equation as in equation
  \ref{eq:x-vector-eigenvalue-equation}, i.e.
  \begin{equation}
    \label{eq:y-vector-eigenvalue-equation}
    \hat{H}_{ss}
  \left | \psi  \right \rangle
  =
  \hbar^{2}
  \left(
    \frac{2}{3} D
    +
    \frac{\sqrt{3}}{\sqrt{2}}E
  \right)
\left | \psi  \right \rangle
  \end{equation}

\item[$\mone$] This case has $ c_{1}=1 $ and $ c_{2}=0 $ since here $ \left |
\psi  \right \rangle = \left | j=1, m=0 \right \rangle  $. Therefore following
the same procedure as in the previous examples we can obtain

\begin{align}
  \label{eq:D_part_spin_spin_general_calculation}
  \hat{H}_{ss}(D)
\left | \psi  \right \rangle &=
  \hbar^{2}D
  \left(0- \frac{1}{3} 2\right)
\left | j_{1}, m_{1} \right \rangle
=
- \frac{2}{3}
\hbar^{2}D
\left | \psi  \right \rangle
\end{align}

Since $ m=0 $ the effect of $ \hat{S}_{\pm} $ on $ \left | \psi  \right \rangle  $ will be zero. Therefore there is only the $ D $ dependent term, i.e.
\begin{equation}
  \label{eq:z-vector-eigenvalue-equation}
  \hat{H}_{ss}
\left | \psi  \right \rangle
=
- \frac{2}{3}
\hbar^{2}D
\left | \psi  \right \rangle
  \end{equation}


\item[$\mmone$]
  In this case we have $ \left | \psi  \right \rangle = \left | j=0, m=0 \right \rangle  $. Analogously to the previous case, the $ E $ term will vanish. Since $ j=0 $ and $ m=0 $ the $ D $ dependent part will also vanish. Thus,
  \begin{equation}
    \label{eq:z-vector-eigenvalue-equation}
    \hat{H}_{ss}
  \left | \psi  \right \rangle
  =
  0
\left | \psi  \right \rangle
  \end{equation}
\end{description}



\end{document}


% vim: spell
