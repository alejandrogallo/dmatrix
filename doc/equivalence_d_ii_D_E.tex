
\documentclass[A4paper, 12pt]{article}

\usepackage[]{amsmath}
\usepackage[margin=0.5cm]{geometry}

% Spins
\newcommand{\uu}{\left | \uparrow\uparrow \right \rangle}
\newcommand{\dd}{\left | \downarrow\downarrow \right \rangle}
\newcommand{\du}{\left | \downarrow\uparrow \right \rangle}
\newcommand{\ud}{\left | \uparrow\downarrow \right \rangle}
\newcommand{\mone}{\frac{1}{\sqrt{2}}\left( \ud + \du\right)}
\newcommand{\mmone}{\frac{1}{\sqrt{2}}\left( \ud - \du\right)}
\newcommand{\x}{\frac{1}{\sqrt 2}\left( \uu + \dd\right)}
\newcommand{\y}{\frac{1}{\sqrt 2}\left( \uu - \dd\right)}

\begin{document}

\section{Equivalence proof}
\label{sec:equivalence_proof}

This is the proof of the equivalence of the two representations of the zero field splitting tensor. Suppose we have the following matrix
\begin{equation}
  D =
  \left(
    \begin{matrix}
      D _{xx} &   &   \\
        & D _{yy} &\\
       && D _{zz}
    \end{matrix}
  \right)
\end{equation}

with the condition that $ \sum_{i} D _{ii} = 0 $ is equivalent to considering the following two
parameters:
\begin{equation}
  E = \frac{D _{yy} - D _{xx}}{2}, \quad D = \frac{3}{2}D _{zz}
\end{equation}

A very quick calculation shows that
\begin{equation}
  \left\{
  \begin{matrix}
    D _{xx} = -E - \frac{1}{3} D \\
    \\
    D _{yy} =  E - \frac{1}{3} D \\
    \\
    D _{zz} = \frac{2}{3} D
  \end{matrix}
  \right .
\end{equation}

which means that the diagonal elements are obtainable from the coefficients $ E $ and $ D $.










\section{Formula deduction}
\label{sec:formula_deduction}

We part from the following Hamiltonian:
\begin{equation}
  \hat{H} _{ss} =
  + D _{x} \hat S ^{2} _{x}
  + D _{y} \hat S ^{2} _{y}
  + D _{z} \hat S ^{2} _{z}
\end{equation}

and for general angular momentum operators we have the following properties and definitions:
\begin{equation}
  \hat{S} _{\pm}  = \hat{S} _{x} \pm i \hat{S} _{y}
\end{equation}
\begin{equation}
  \left\{\hat{S} _{+} , \hat{S} _{-}  \right\} = 2(\mathbf{S} ^{2} - \hat{S} _{z} ^{2} )
\end{equation}

Let us replace in the expression for $ \hat{H} _{ss} $ the coefficients $ D _{i}  $ by
$ E $ and $ D $ , i.e:


\begin{align*}
  \hat{H} _{ss} &=
  - \left(E + \frac{1}{3}D\right)  \hat S ^{2} _{x}
  + \left(E - \frac{1}{3}D\right)  \hat S ^{2} _{y}
  + \frac{2}{3}D \hat S ^{2} _{z} \\
  &=
  E \left( \hat{S} _{y} ^{2}  -\hat{S} _{x} ^{2}  \right)
  +
  \frac{D}{3} \left[ -\hat{S} ^{2} _{x} -\hat{S} _{y} ^{2} + 2\hat{S} _{z} ^{2}  \right]
\end{align*}

now we can express of course

\begin{align*}
  -\hat{S} ^{2} _{x} -\hat{S} _{y} ^{2} + 2\hat{S} _{z} ^{2} &=
  \hat{S} ^{2} _{z} -\mathbf S^{2} + 2\hat{S} _{z} ^{2} =
  3\hat{S} ^{2} _{z} -\mathbf S^{2}
\end{align*}
With this we can express the term of $ \hat{H} _{ss}   $ depending on $ D $ as
\begin{equation}
  D
  \left(
    \hat{S} _{z} ^{2} - \frac{1}{3} \mathbf{S} ^{2}.
  \right)
\end{equation}

On the other hand we have the term $ \hat{S} _{y} ^{2} -\hat{S} _{x} ^{2}  $
which we can rewrite as
\begin{equation}
  \mathbf{S} ^{2} - \hat{S}_{z}^{2} - 2 \hat{S}_{x}^{2}
  =
  \frac{1}{2}\left\{\hat{S} _{+} , \hat{S} _{-}  \right\}
  - \frac{2}{4}\left( \hat{S} _{+}  + \hat{S} _{-}   \right) ^{2}
  =
  -\frac{\hat{S} _{+} ^{2} + \hat{S} _{-} ^{2}}{2}
\end{equation}

With all we can write for the Spin-Spin Hamilton operator
\begin{equation}
  \hat{H} _{ss}
  =
  \frac{-E}{2}\left( \hat{S} _{+} ^{2} + \hat{S} _{-} ^{2}   \right)
  +
  D
  \left(
    \hat{S} _{z} ^{2} - \frac{1}{3} \mathbf{S}^{2}.
  \right)
\end{equation}





\section{Eigenvectors}
\label{sec:eigenvectors}

The eigenvectors of the operator $ \hat{H} _{ss}   $ are not anymore the common eigenbasis of $ \hat{S}_{z} $ and $ \mathbf{S}^{2}  $, i.e,
\begin{equation}
  \left\{
    \uu, \dd, \mone, \mmone
  \right\}
\end{equation}
This is so because of the states with quantun number $ m_S \neq 0 $ , for example one sees that
\begin{equation}
  \hat{H} _{ss} \uu = a\uu + b \dd
\end{equation}
with $ a $ and $ b $ different from zero.
Instead, the new eigenbasis is formed by the following vectors:
\begin{equation}
  \left\{\mone, \x, \y, \mmone \right\}
\end{equation}

In order to see that this is indeed the case, one has to know how each operator appearing in $ \hat{H} _{ss}  $  acts on every state.

In general we have for an angular momentum operator $ \hat{J}  $ that
\begin{equation}
  \hat{J} _{\pm} \left | j, m \right \rangle = c _{\pm} \left | j,m \pm 1 \right \rangle
\end{equation}
where
\begin{equation}
  c _{\pm}  = \hbar \sqrt{(j\mp m)(j \pm m +1)}
\end{equation}

where $ j $ and $ m $ are the pertinent quantum numbers. In our case $ j \in \left \{0,1 \right \}  $.
We can calculate the eigenvalues for $ \uu $ and $ \dd $, we need specially the following result
\begin{align*}
  \hat{S} _{-} ^{2} \uu &=
  \hat{S} _{-} ^{2} \left | j=1, m=1 \right \rangle =
  \sqrt{2\cdot 1} \hbar \hat{S} _{-}  \left | j=1, m=0 \right \rangle = \\
  &
  \sqrt{1\cdot 2} \hbar  \sqrt{2\cdot 1} \hbar  \left | j=1, m=-1 \right \rangle =
  2\hbar ^{2} \dd
\end{align*}
and in the same spirit one can calculate $ \hat{S} _{+} ^{2} \dd = 2 \hbar ^{2}  $.
Now we can calculate the eigenvalues:
\begin{equation}
  \hat{H} _{ss} \left( \uu \pm \dd \right)
  =
  \left(
    \frac{D}{3}\pm (-E)\hbar
  \right)
  \left(
    \uu \pm \dd
  \right)
\end{equation}

Since $ \mmone $ has quantum numbers $ j=0 $ and $ m=0 $ we have naturally
\begin{equation*}
  \hat{H} _{ss} \left( \mmone \right) = 0
\end{equation*}
and of course for the case $ j=1 $ and $ m=0 $ we have
\begin{equation}
  \hat{H} _{ss} \left( \mone \right) = -\frac{2}{3}D \left( \mone \right)
\end{equation}

Therefore, in this basis the Hamilton operator has the following Matrix form

\begin{equation}
  \begin{pmatrix}
    \frac{D}{3} - E & & & \\
                 &\frac{D}{3} + E & & \\
               & & -\frac{2}{3}D & \\
               &&&0
  \end{pmatrix}
  \begin{matrix}
    \x\\
    \y\\
    \mone\\
    \mmone
  \end{matrix}
\end{equation}


\end{document}
